\documentclass [10pt]{article}

%* some useful packages *********
\usepackage{amsfonts,amssymb}
\usepackage{epsfig,natbib,float}
\usepackage[ansinew]{inputenc}
\usepackage{graphicx}
\usepackage{booktabs}

%****************************************************************
\setlength{\oddsidemargin}{+4.6mm}
%
\setlength{\textwidth}{15cm}
%
\setlength{\textheight}{23cm}
%
\setlength{\topmargin}{-1.25cm} \setlength{\baselineskip}{1mm}
%
\setlength{\parindent}{0pt}
%
\setlength{\parskip}{0.25cm}
%
\pagestyle{empty}

%*for Spanish*************************************************
\newcommand{\enye}{\~n}
%**************************************************************


\usepackage{fancyhdr}
\pagestyle{fancy}
\lhead{This is my name}
\rhead{this is page \thepage}
\cfoot{CoDaWork 2017 --- Abbadia San Salvatore (IT)}
\renewcommand{\headrulewidth}{0.4pt}
\renewcommand{\footrulewidth}{0.4pt}


%****************the document **************************

\begin{document}
\begin{center}
\textbf{\large Title: Instructions for authors - CoDaWork 2017}

\bigskip
\textbf{A.N. Other}$^{1}$\textbf{, U.T. Cobley}$^{2}$\textbf{, and S.F. Adams}$^{3}$ \\
{\small
$^{1}$Bloggs University, Biggleswade, UK; \textit{an.other@hotmail.com} \\
$^{2}$Institute of Mathematical Geology, St.Petersburg, Russia \\
$^{3}$PB Petroleum, Houtson, Texas, USA}
\end{center}

\bigskip
{\centerline{\bf Abstract}}

\begin{quote}
Write an informative abstract, not longer than 300 words. The abstract paragraph should be indented about 0.25~inches on both left and right-hand margins.

\medskip

{\bf Key words: } Include some informative key words. \par
\end{quote}

\newpage

\section{Introduction}
This is the introduction, starting in a new page. The paper size should be set to A4 (not American quarto).
Margins should be 30 mm on left and right, 25 mm at top and bottom. Paragraphs should be separated by an empty line in
this way. Text should be justified.

Remember that the final electronic paper --without limit in the number of pages-- will be due {\bf before May 4,
2017}. It will be sent in PDF format. It will be published in the CoDaWork'17 CD (with ISBN) and will be available in
the sessions of the workshop. {\bf Only papers of participants registered before May 4, 2017 will be included in the
proceedings CD}.

\section{First level headings}
The workshop on Compositional Data is intended as a forum for discussion of important points related to the statistical treatment and modelling of compositional data, as well as their applications and interpretation. The goal of such
discussions is to get some insight into the most appealing future lines of research in the field.

\subsection{Second level headings}
In order to meet this general but clear goal, we intend to bring together a significant number of specialists, users and interested people to collect critical contributions and start a stimulating brainstorming.

\subsubsection{Third level headings}
The Introductory course to statistical analysis of compositional data will work from a variety of practical compositonal problems. Different case studies will be presented and analyzed using CoDaPack, a freeware software based on EXCEL. This software is oriented to users coming from the applied sciences. No extensive background in using computer packages is
required.


\section{Citations, figures and references}
\subsection{Citations in text}
Citations within the text should include the author's last name and year: ``The air conditioner data (Proschan,
1963) ...'', or when the author is used as a noun in the sentence: ``Proschan (1963) presented a data set ...''.

In text, captions, and table headings, list all authors if two or fewer, and just the first author followed by ``et al." for more. Examples:

\begin{center}
(Jones and Johnson, 1986; Emmanuel et al., 1989)
\end{center}
or
\begin{center}
Emmanuel et al. (1989) showed that ..., whereas Jones and Johnson (1986) found that ...
\end{center}

When giving a quote or referring to a specific fact or formula in a book or from an article of more than 8 pages, the citation should include the page number. Examples:

\begin{center}
(Chayes, 1956, p. 55) or (Matheron, 1975, p. 229).
\end{center}

Page numbers should not be given in the text when referring to the work as a whole. As with figures, you do not need to direct the reader to ``see" a citation to the literature. Be sure your references are accurate and formatted correctly.

\subsection{Figures}
All figures should be inserted within the text exactly as they should appear when printed. All figures must be
centered. Figure number and caption always appear below the figure.

\begin{figure}[ht]
\begin{center}
x x x x x
\end{center}
\caption{This is a figure caption.}
\end{figure}

When you refer to an illustration, capitalize and spell out the word ``Figure'' if not in parenthesis, as in \ \ ``Figure 2 shows that the distribution of permeability is skewed ...''; or abbreviate if in parenthesis, as in \ \  ``The distribution of permeability is skewed (Fig. 2) ...''.

If you have multiple parts in a figure, then label them with capital letters A,B,C, etc. Refer to them in the text as Figure 2A, or (Fig. 2A). In captions, follow this example:

\begin{figure}[ht]
\begin{center}
\begin{center}
x x x x x
\end{center}
\end{center}
\caption{Density functions: (A) Permeability; (B) Porosity.}
\end{figure}

\subsection{Tables}
All tables must be centered, neat, clean, and legible. Table number and title always appear above the table (see the example below).

\begin{table}[ht]
\caption{This is an example of a table.}
\begin{center}
\begin{tabular}{lr}
\toprule
Income         &$\$ 42.94$ \\ Expenses       &$\$ 26.12$ \\ \midrule
Rest           &$\$ 16.82$ \\ \bottomrule
\end{tabular}
\end{center}
\end{table}

The word ``Table'' should be always capitalized, and not abbreviated even in parentheses.

\subsection{Equations}
The word ``Equation'' should be always capitalized and spelled out in the text, as in ``It follows from Equation (3) that ..'' but capitalized and abbreviated in parenthesis, as in ``It follows [Eq. (3)] that ...''. If you use any other word to refer to an equation, such as ``expression'' or ``relationship'', do not capitalize.

\section*{Acknowledgements and appendices}
Use non-numbered first level headings for the acknowledgements. They should follow text, and precede the list of references. Appendices follow references, and should be headed ``{\bf Appendix} A'' etc. if more than one.

\section*{References}
References follow the acknowledgements. Use first level no-numbered headings. The bibliography should follow the {\it Chicago style}. See below for examples.

\hangindent=0.5cm \hangafter=1 %
Ghahramani, Z. (1997).
\newblock Learning dynamic Bayesian networks.
\newblock In C.~Giles and M.~Gori (Eds.), {\em Adaptive Processing of Sequences and Data Structures}, Lecture Notes in Artificial Intelligence, pp.\ 168--197. Berlin: Springer Verlag.

\hangindent=0.5cm \hangafter=1 %
Pearl, J. (1986).
\newblock Fusion propagation and structuring in belief networks.
\newblock {\em Artificial Intelligence\/}~{\em 29\/}(3), pp. 241--288.

\hangindent=0.5cm \hangafter=1 %
Whittaker, J. (1990).
\newblock {\em Graphical models in applied multivariate statistics}.
\newblock Chichester: Wiley.

If you use BibTex, use (see the LaTeX file):

\bibliographystyle{chicago}
\bibliography{name-of-your-BibTex-file}


\end{document}